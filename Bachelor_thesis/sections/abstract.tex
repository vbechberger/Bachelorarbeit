\renewenvironment{abstract}
{\vspace*{\fill}
  {\bfseries\abstractname}\\[5mm]}
{\vfill}

\selectlanguage{ngerman}
\begin{abstract}
  Diese Bachelorarbeit wendet genetische Algorithmen auf das Traveling Salesman Problem an. Das Traveling Salesman Problem (TSP) ist ein NP-schweres Optimierungsproblem, bei dem die kürzeste Rundtour durch eine gegebene Menge an Städten gesucht wird. Genetische Algorithmen sind evolutionäre Algorithmen, die vom Prozess der natürlichen Selektion inspiriert sind. Diese Bachelorarbeit gibt einen Überblick über verschiedene Varianten genetischer Algorithmen im Zusammenhang mit dem TSP. Sie beschreibt die wichtigsten Repräsentationstypen für Rundtouren und diskutiert jeweils verschiedene Selektions-, Crossover- und Mutations-Operatoren. Außerdem analysiert diese Bachelorarbeit experimentell, wie effektiv genetische Algorithmen das TSP lösen. Zunächst werden die generellen Hyperparameter Populationsgröße und Mutationsrate optimiert. Im Hauptteil der Experimente werden dann die Ergebnisse verschiedener Repräsentationstypen sowie verschiedener Kombinationen von Crossover- und Mutationsoperatoren verglichen. Abschließend listet diese Bachelorarbeit die Operatoren mit der besten und schlechtesten Performance und vergleicht die beobachteten Ergebnisse mit den Effekten, die in der Literatur beschrieben wurden.
\end{abstract}

\selectlanguage{english}
\begin{abstract}
  This bachelor thesis applies genetic algorithms to the traveling salesman problem. The traveling salesman problem (TSP) is an NP-hard optimization problem which looks for a shortest round tour through a given set of cities. Genetic algorithms are evolutionary algorithms inspired by the process of natural selection. This thesis provides an overview of different variations of genetic algorithms for solving the TSP. It describes the main approaches for representing a round tour and discusses \mbox{different} variants of selection, crossover and mutation operators, respectively. Moreover, this thesis \mbox{analyzes} experimentally the effectiveness of genetic algorithms with respect to solving the TSP. First, the general hyperparameters of population size and mutation rate are optimized. In the main part of the experiments, the performance of different representation types and combinations of different crossover and mutation operators are compared to each other. Finally, this thesis reports the best and the worst performing operators and compares the observed results to effects described in the literature.
\end{abstract} 
