\section{Introduction}
\label{sec:introduction}

The traveling salesman problem (TSP) is a well-known optimization problem, where one aims to identify a shortest round tour through a given set of cities, provided that each city must be visited exactly once. Despite the fact that the problem can be easily formulated, it is known to be NP-hard. This means that an exact solution cannot be found in polynomial time unless P = NP. Applying exact algorithms therefore requires enormous computing efforts. Nevertheless, the TSP is widely studied in the field of combinatorial optimization due to its great theoretical and practical importance. On the one hand, as one of the standard optimization problems it serves as a basis for studying general optimization methods. Successful methods can then be transferred to other optimization problems. On the other hand, the TSP can be applied to a wide variety of real-world problems in different fields which include of course transportation and logistics applications. The simplicity of the problem definition has however also caused many interesting applications in other areas, such as genome sequencing, scan chains, drilling problems, aiming telescopes and X-rays, data clustering, and locating power cables (see \cite{applegate2006traveling}). \\

There exist many methods to solve the TSP. However, being NP-hard, this problem requires a compromise between the quality of solutions and processing time. Different approximation techniques can be applied to find a reasonably good but not necessarily optimal solution. One of those techniques are so-called genetic algorithms. \par 

Genetic algorithms are search and optimization algorithms inspired by natural selection and genetics. A genetic algorithm includes several common steps, including selection, crossover, and mutation. However, these steps can vary greatly among different approaches in their formulation. This gives rise to plenty of modifications and a great variability for this type of optimization method. \par

This bachelor thesis considers genetic algorithms as an optimization method for solving the TSP. The purpose of this thesis is twofold: On the one hand, it gives a detailed theoretical overview of genetic algorithms in context of the TSP. This overview is mainly based on the paper “Genetic algorithms for the traveling salesman problem” \cite{potvin1996genetic} by Jean-Yves Potvin. On the other hand, this thesis provides a unified implementation of genetic algorithms for the TSP. This implementation is used for a comparative analysis of different variations of genetic algorithms for solving the TSP. The existing literature in this field in general comes without an implementation and typically leaves many hyperparameters unspecified. Even if they are specified, "the results cannot be easily compared because various algorithmic designs and parameter setting are used" \cite{potvin1996genetic}. Therefore, our analysis can provide additional insight for comparing different \mbox{proposals}.\\ 

The remainder of this thesis is structured as follows: Chapter \ref{sec:general_definitions} gives some background with respect to the TSP and genetic algorithms. Chapter \ref{sec:representation_types} summarizes different types for representing candidate solutions and specifies which ones were used for the current research. Chapter \ref{sec:heuristics} introduces several heuristics which were used to construct the initial population in the experiments. Chapters \ref{sec:selection}, \ref{sec:crossover}, and \ref{sec:mutation} give an overview of the existing selection, crossover, and mutation operators, respectively. After that, Chapter \ref{sec:prep} reports experiments for finding good settings for the general parameters of genetic algorithms. Chapter \ref{exp_crossovers} then contains a comparative study of different combinations of crossover and mutation operators for the individual representation types. Chapter \ref{exp_repr_types} compares the effectiveness of different approaches for representing candidate solutions. Moreover, it reports the experiments for the operators which have shown the best performance. Finally, Chapter \ref{sec:conclusion} concludes this thesis by summarizing the main findings and giving an outlook on future work.\par
