\section{Conclusion}
\label{sec:conclusion}

In the context of this bachelor thesis, a detailed theoretical overview of genetic algorithms in context of the TSP was provided. After that, we have analyzed different combinations of crossover and mutation operators and tuned the parameters for them to find a best variation of a genetic algorithm for solving the TSP. This analysis was based on the results of the experiments which were conducted on about one hundred TSP instances of different size.\par 

The search for the best hyperparameters has shown that a best configuration is very sensitive to the operators used and the size of the TSP instance. Our approach of taking into account the length of the tour when determining the size of the population inspired by the investigations of \citeauthor{chen2015measuring} \cite{chen2015measuring} has not shown any promising results. On average, a population size fixed at 100 chromosomes and a mutation rate of $10\%$ yielded the best results. \par 

Concerning the statement made by \citeauthor{potvin1996genetic} \cite{potvin1996genetic} about the path representation crossovers, we can confirm that the crossover operators which preserve relative order perform much better than the crossover operators which preserve absolute order only if the OBX is not taken into consideration.\par 

For small instances with less than 100 cities, our experiments confirm a great performance of the ERX in comparison to other path crossover operators which was shown in studies made by \citeauthor{starkweather1991comparison} \cite{starkweather1991comparison}. However, we can not confirm the statement by \citeauthor{potvin1996genetic} \cite{potvin1996genetic} that "the edge-preserving operators are superior to the other types of crossover operators". On our dataset, no representation type has performed constantly better or worse than the other. Each group of crossover operators has some operators which have performed poorly and has the ones which have shown good results: The OBX was the worst in the group of the path representation crossover operators while the AEX was the worst operator which uses the adjacency representation type. The ordinal representation has not performed worst. However, it has neither shown the best results. This does not support the point of view by \citeauthor{potvin1996genetic} \cite{potvin1996genetic} that "this representation is mostly of historic interest".\par

Concerning the six mutation operators, the inversion mutation has drastically outperformed all the other mutation types in all combinations and for all instances. This confirmed the great importance of this mutation operator declared by \citeauthor{potvin1996genetic} \cite{potvin1996genetic}. Among the mutation operators which have shown the worst results, the scramble mutation can be named.\par 

The winner among the crossover operators depends on the way the population is initialized: If only random permutations of cities are used, the HX performs best. If the population is initialized using results of construction heuristics, the MX performs best. The OX crossover has not shown the best results, but it constantly performed well which confirms the results from the literature (see \cite{starkweather1991comparison}).\par 

The tournament selection performs better if the number of participants is drawn randomly from some interval and not constantly fixed at one number.\par 

Concerning the general performance, our genetic algorithm with random initialization has outperformed the results of four construction heuristics for instances with less than 200 cities. For larger instances, using only the random permutations has performed poorly. \par 

Using the heuristics for population initialization has drastically increased the performance and has shown better results for smaller instances than for the larger ones as well. In fact, the four construction heuristics we used sometimes returned the optimal solution for smaller instances; therefore, using genetic algorithm can be unnecessary in this case. \par 

The fact that the performance of our genetic algorithm was better for the smaller instances than for the larger ones independent from the type of the initialization of the population has proven our point that the length of the tour influences the results. A drastic performance decrease with an increasing problem size in all experiments can be easily explained by the fact that our genetic algorithm has made much more iterations for smaller instances than for the larger ones. As a result, the approach to choose a time limit as a stop criterion is rather questionable if instances of real world applications have to be considered. \par 

Taking into account the fact that our genetic algorithm has made much less iterations for the larger instances, we cannot really evaluate the statement by \citeauthor{potvin1996genetic} \cite{potvin1996genetic} that the edge preserving operators (AEX, ERX, and HX) cannot find good solutions for larger TSP instances. Operators of all representation types performed worse in our experiments in this case. In order to evaluate this statement, a fair stop condition like a number of iterations has to be used. Therefore, as the directions for future work, increasing the time limit or using another stop criterion such as number of iterations can be considered. \par 

In our research, we have used four simplest construction heuristics. One can consider other heuristics (e.g., savings heuristic) which can be used to initialize the population. Moreover, the ordinal representation has not shown the worst results; therefore this representation type should not be completely excluded from consideration. Developing different crossover and mutation operators specifically for it could increase its performance. The fact that this representation type always produces valid offsprings without making additional steps to ensure the validity, makes it applicable also to other optimization problems. Finally, concerning the adjacency representation, some special mutation operators which aim at preserving the edges from the fittest parents can be developed for it.\par 




